\documentclass{article}
\usepackage{listings}
\usepackage{amsmath}
\usepackage{graphicx}
\usepackage{float}
\usepackage{subcaption}
\usepackage[linewidth=1pt]{mdframed}
\usepackage[colorlinks]{hyperref}

\usepackage{verbatim}

\usepackage{algorithm}
\usepackage{algpseudocode}

\hypersetup{citecolor=DeepPink4}
\hypersetup{linkcolor=DarkRed}
\hypersetup{urlcolor=Blue}

\usepackage{cleveref}

\setlength{\parindent}{1em}
\setlength{\parskip}{1em}
\renewcommand{\baselinestretch}{1.0}

\begin{document}

\begin{titlepage}
	\centering
	{\scshape\LARGE Assignment 4\par}
	\vspace{1cm}
	{\scshape\Large Algorithms \& Complexity (CIS 522-01)\par}
	\vspace{1.5cm}
	{\Large\itshape Javier Arechalde\par}
	\vfill
	{\large \today\par}
\end{titlepage}

\section*{Part A: Read the solved exercises and Practice}

\subsection*{Solved excercise \#1 in Chapter 5}

In this problem, we are given an array containing $n$ entries. Inside this array, we haev a peak entry $p$ in a position $j$ of the array, so that the array positions prior to $p$ go in increasing order until we reach $p$, and the values after $p$ go in straight decreasing order.

Our goal is to find that peak entry $p$ without having to read the entire array, and only by reading as few values as possible. 

In our algorithm, we will first check the element in the middle of the given array, and we will check if:

\begin{itemize}
 \item $array[n/2-1]<array[n/2]<array[n/2]+1$
 \item $array[n/2-1]>array[n/2]>array[n/2]+1$
 \item $array[n/2-1]<array[n/2]>array[n/2]+1$
\end{itemize}

In the first case, we are in a positive slope, which means that we still didn't reach the $p$ value, in this case, we will get rid of the left half of the array and continue. In the second case, we are in a negative slope, which means that we already passed the $p$ value, in this case, we will get rid of the right half of the array and continue. In the last case, the value is greater than the value that goes before it, and greater than the value that goes after it, which means that we hit the $p$ value, in this case, we will stop running our algorithm and return $p$.

We will follow this structure recursively, checking the value in the middle and getting rid of one of the halves of the array until we find the $p$ value.

\subsubsection*{Algorithm Pseudocode}

\begin{algorithm}[H]
\caption{Finding maximum pseudocode}
\begin{algorithmic}[1]
\Function{Findmaximum}{$pos_{start},pos_{end},array$}
 \State $n = (pos_{start}+pos_{end})/(2)$
 \If{$array(\frac{n}{2}-1)<array(\frac{n}{2})<array(\frac{n}{2}+1)$}
  \State We have a \textbf{positive slope}, so we havent reached the maximum yet
  \State \Call{Findmaximum}{$(pos_{start}+pos_{end})/2, pos_{end}, array$}
 \ElsIf{$array(\frac{n}{2}-1)>array(\frac{n}{2})>array(\frac{n}{2}+1)$}
  \State We have a \textbf{negative slope}, we already passed the maximum
  \State \Call{Findmaximum}{$pos_{start},(pos_{start}+pos_{end})/2$}
 \ElsIf{$array(\frac{n}{2}-1)<array(n/2)>array(\frac{n}{2}+1)$}
  \State We have found the \textbf{maximum point}
  \State \Return value(n/2)
 \EndIf
\EndFunction
\end{algorithmic}
\end{algorithm}

\subsubsection*{Solution for problem instance of size $10$}

Now we are going to prove that our algorithm works, by manually running it over a sample set of size $10$. This will be our problem's sample set:

$$S = [1,2,4,12,14,21,6,4,3,1]$$

When we start running the algorithm, we will first check the number in the $5^{th}$ position.

In this case we will have that $S[4] = 12<S[5] = 14<S[6] = 21$, which means that we are in a positive slope, and we still haven't reached the peak value $p$. Then we will call the function again, so in the next iteration of our algorithm, we will work with the set that is on the right side of the value that we checked.

$$S' = [14,21,6,4,3,1]$$

Now we will check the $3^{rd}$ position. In this case we have that $S[2] = 21>S[3] = 6>S[4] = 4$, which means that we are in a negative slope, so we already passed the peak value $p$. Thus, we will call the function again, and in the next iteration of our algorithm we will work with the set at the left of the value we just checked.

$$S'' = [14,21,6]$$

Now we will check the middle position, in this case the $2^{nd}$ position. We get the result that $S[1] = 14<S[2] = 21>S[3] = 6$, which means that the number we are checking is indeed the peak entry $p$. Now we will return that value, and stop running our algorithm.

$$max = 21$$

\subsubsection*{Time Complexity}

In this problem, with each one of the recursive calls, we reduce the problem to one of at most half the size of the initial problem. Then,

$$T(n)\leq T(n/2) + c$$

when $n>2$, and

$$T(2)\leq c$$

Then, we can confirm that the running time of our algorithm will be $O(\log n)$.

\pagebreak

\subsection*{Solved exercise \#2 in Chapter 5}

In this problem, we have an investment company that stores during $n$ consecutive days the values of a given stock. For each of these days, the stock has a price $p(i)$ per share for the stock on that day. We assume that the stock prize was fixed on that day.

The goal is to find, without having to check each possible combination of days, which will take $O(n^2)$, in which day they should have bought the shares, and in which day they must have sold them to make as much money as possible.

In our algorithm, we will recursively divide our problem while our array is longer than 2, until we are left with two sets of $S$ and $S'$. Then we will substract the first value to the second value of the array, this way we calculate the potential benefit that could be earned if buying in the first day and selling on the second. We will also calculate the optimal solution of substracting the minimum value on $S$ to the maximum value of $S'$. In the end we will compare the three options, to see which one yields more benefit, and return that option.

\begin{itemize}
 \item The optimal solution on $S$
 \item The optimal solution on $S'$
 \item the optimal solution of $p(j)-p(i)$, over $i \in S$ and $j \in S'$.
\end{itemize}

\subsubsection*{Algorithm Pseudocode}

\begin{algorithm}[H]
\caption{Stocks Divide-and-Conquer pseudocode}
\begin{algorithmic}[1]
\Function{Max}{list}
 \State $max = list[0]$
 \For{element in list}
  \If{$element>max$}
   \State max = element
  \Else{}
   \State \textbf{continue}
  \EndIf
 \EndFor
 \State \Return max
\EndFunction
\Function{Min}{list}
 \State $min = list[0]$
 \For{element in list}
  \If{$element<min$}
   \State min = element
  \Else{}
   \State \textbf{continue}
  \EndIf
 \EndFor
 \State \Return min
\EndFunction
\Function{FindOpt}{$array$}
\If{$len(array)>2$}
 \State We divide the array in two halves $Larray$ and $Rarray$
 \State LSide = \Call{FindOpt}{$Larray$}
 \State RSide = \Call{FindOpt}{$Rarray$}
 \State LOpt = LSide[1] - LSide[0]
 \State ROpt = RSide[1] - RSide[0]
 \State MOpt = \Call{Max}{RSide} - \Call{Min}{LSide}
 \State Marray = [\Call{Min}{LSide},\Call{Max}{RSide}]
 \State maxvalue = \Call{Max}{LOpt,ROpt,MOpt}

 \If {$maxvalue == LOpt$}
  \State \Return LSide
 \ElsIf{$maxvalue == ROpt$}
  \State \Return RSide
 \ElsIf{$maxvalue == MOpt$}
  \State \Return Marray
 \EndIf
\ElsIf{$len(array) <= 2$}
 \State \Return array
\EndIf
\EndFunction
\end{algorithmic}
\end{algorithm}

\subsubsection*{Solution for problem instance of size $10$}

Now we are going to prove that our algorithm works, by manually running it over a sample set of size $10$. This will be our problem's sample set:

$$S = [7,4,3,2,5,7,10,20,14,7]$$

First, as the array is longer than 2, we will divide it into two sets recursively until we have arrays of length smaller or equal to 2.

$$[7,4,3,2,5] \quad [7,10,20,14,7]$$

$$[7,4] \quad [3,2,5] \quad [7,10] \quad [20,14,7]$$

$$[7,4] \quad [3,2] \quad [5] \quad [7,10] \quad [20,14] \quad [7]$$

Now we are going to compare set by set, if the optimal solution between one set, compare it with the other set, and compare it also with the maximum of the "right" set and the minimum of the "left" set, and see which one gives us a better solution.

Now we are going to compare the sets, to see which is the best solution, between $S$, $S'$ or $p(j)-p(i)$, over $i \in S$ and $j \in S'$. We will return the best solution in each iteration.

$$[7,4] \quad \& \quad [3,2]$$

$$OptL = 4-7 = -3 \quad OptR = 2-3 = -1 \quad OptM = 3-4 = -1$$

$$[5] \quad \& \quad [7,10]$$

$$OptM = 10-5 = 5 \quad OptR = 3$$

$$[20,14] \quad  \& \quad [7]$$

$$OptL = -6 \quad OptM = 7-14 = -7$$

We return $[3,2] \quad [5,10] \quad [20,14]$.

Now we have:

$$[3,2] \quad [5,10] \quad [20,14]$$

Lets start comparing.

$$[3,2] \quad \& \quad [5,10]$$

$$OptL = -1 \quad OptR = 5 \quad OptM = 10 - 2 = 8$$

We return $[2,10]$.

Now we have:

$$[2,10] \quad \& \quad [20,14]$$

$$OptL = 8 \quad OptR = -6 \quad OptM = 18$$

We finally return $[2,20]$, and that will be our final result.

\subsubsection*{Time Complexity}

As we said, in this problem, we will have to recursively subdivide our array, and take the best possible solution out of this three possible solutions:

\begin{itemize}
 \item The optimal solution on $S$
 \item The optimal solution on $S'$
 \item the optimal solution of $p(j)-p(i)$, over $i \in S$ and $j \in S'$.
\end{itemize}

The first two items are computed, in time $T(n/2)$ by recursion, and the third item is computed by finding the maximum in $S'$ and the minimum in $S$, which can be done in $O(n)$ time. Then our running time $T(n)$ satisfies

$$T(n) \leq 2T(\frac{n}{2})+O(n)$$,

so the time complexity of our implementation will be $O(n\log n)$.


\section*{Part B: Problem Solving}

\subsection*{Significant inversion}

\subsubsection*{Problem Model}

In this problem, we are given a sequence of $n$ numbers $a_1,...,a_n$, which we will we assume that are all distinct, and we define inversion to be a pair $i<j$ such that $a_i>a_j$. We call a pair \textit{significant inversion} if $i<j$ and $a_i<2a_j$.

Our goal is to count the number of \textit{significant inversions} between two orderings, using an algorithm that has $O(n \log n)$ time complexity.

Our implementation will be based on the \textit{merge-sort} algorithm, as we will recursively sort and merge the array, but we will also count the number of significant inversions while at it.

We also added some improvements to the pseudocode. If for example, while we are merging the two sorted arrays, we finish adding elements from one array while there are still elements available in the other array, we will add this remaining elements and exit the while loop.

Another important thing to note is that when we are merging two arrays and checking if there exists a significant inversion, if there is indeed one, there will be as many significant inversions as elements are left in the array on the "left", as the array is sorted in increasing order, then the rest of the values that come after that one will also be greater than the value we are comparing with.

\subsubsection*{Pseudocode}

\begin{algorithm}[H]
\caption{Significant inversion pseudocode}
\begin{algorithmic}[1]
\Function{CountInversions}{array}
 \If{$length(array)>2$}
  \State Divide the array in two halves $array_{left}$ and $array_{right}$ and
  \State recursively call this function
  \State $array_{left} = \Call{CountInversions}{array_{left}}$
  \State $array_{right} = \Call{CountInversions}{array_{right}}$
  \While{We havent sorted $array_{left}$ and $array_{right}$}
   \State $value_{left}  = array_{left}[i]$ and $value_{right} = array_{right}[k]$
   \If{We added all the elements in $array_{left}$}
    \State $array_{sorted} = array_{sorted} + array_{right}[j:end]$
    \State Break loop
   \ElsIf{We added all the elements in $array_{right}$}
    \State $array_{sorted} = array_{sorted} + array_{left}[i:end]$
    \State Break loop
   \EndIf
   \If{$value_{left}<value_{right}$}
    \State Add $value_{left}$ to the end of $array_{sorted}$
    \State i++
   \Else
    \State Add $value_{right}$ to the end of $array_{sorted}$
    \State j++
    \If{$value_{left}>2*value_{right}$}
     \State All the elements from position $i$ to the end will be greater
     \State too as $array_{left}$ is sorted in increasing order
     \State $N_{inv} = N_{inv} + length(array_{left}) - i$
    \EndIf
   \EndIf
  \EndWhile
  \State \Return $array_{sorted}$
 \EndIf
 \If{$length(array) == 1$}
  \State \Return array
 \ElsIf
  \If{$array[0]<array[1]$}
   \State \Return $array[0],array[1]$
  \Else
   \If{$array[0]>2*array[1]$}
    \State $N_{inv}++$
   \EndIf
   \State \Return $array[1],array[0]$
  \EndIf
 \EndIf
\EndFunction
\end{algorithmic}
\end{algorithm}

\subsubsection*{Implementation}

Here is the code for the implementation of the \textit{pseudocode} shown above.

\lstinputlisting[language=Python]{Problem52.py}

\subsubsection*{Running time}

Our algorithm uses mergesort to sort the array and count the number of significant inversions at the same time, so the running time of our algorithm will be $O(n\log n)$

\subsection*{Local minimum}

\subsubsection*{Problem Model}

In this problem, we are give a complete binary tree $T$. They also state that the number of nodes is $n = 2^d -1$ for some $d$, which means that the tree is not only complete, but is also perfect, which means that all nodes that have childre, have two children, and all leaves are at the same level. This is extremely important, as if the tree was only complete, instead of perfect, we wont have all the leaves at the same level, so we would have to come up with a method to avoid trying to go deeper into the tree when we were reaching the last level, which is not completely filled and all the leaves are as left as possible.


Each node $v$ of $T$ is labeled with a real number $x_v$. For each node in the tree, we can only determine its value $x_v$ by probing the node $v$.

Our goal is to find a local minimum, that is if the label $x_v$ is less than the label $x_w$ for all the nodes $w$ that are joined to $v$ by an edge.  We also have to find this local minimum of $T$ using only $O(log n)$ probes to the nodes of $T$.

In our proposal, we will recursively go deeper into the tree until we reach the leave nodes, once we reach the leave nodes, we will compare the one on the left and the one on the right and return the smaller one, while we go up until the tree root node. This can be seen as a "tournament", as only the smaller value will remain after every round, and in the end we will have the smallest value in the tree.

To explore the tree, we will use a position array called $pos_{array}$ that we will initialize as $\{0\}$ and when we go deeper in the tree, we will append in each level, a $0$ if we are exploring the left branch and $1$ if we are exploring the right branch. Then when we want to probe a node, we will use this array to reference it. For example, if the tree has $d = 3$ and we want to reference the leftmost leaf, we will do $x_{v} = T(\{0,0,0\})$

\subsubsection*{Pseudocode}

\begin{algorithm}[H]
\caption{Local minimum pseudocode}
\begin{algorithmic}[1]
\Function{FindMin}{$T,level,pos_{array}$}
 \If{We haven't reached the leaves of T}
  \State We go down one level in the tree
  \State $left_{val} = \Call{FindMin}{T,level+1,pos_{array}.append(0)}$
  \State $right_{val} = \Call{FindMin}{T,level+1,pos_{array}.append(1)}$
  \State
  \State We find the minimum value and return it
  \If{$left_{val}<right_{val}$}
   \State \Return $left_{val}$
  \Else
   \State \Return $right_{val}$
  \EndIf
 \EndIf

 \If{We are on a leaf}
  \State \Return $x_v = T(pos_{array})$
 \EndIf
\EndFunction
\end{algorithmic}
\end{algorithm}

\subsubsection*{Running time}

In this problem, with each one of the recursive calls, we reduce the problem to one of at most half the size of the initial problem. Then,

$$T(n)\leq T(n/2) + c$$

when $n>2$, and

$$T(2)\leq c$$

Then, we can confirm that the running time of our algorithm will be $O(\log n)$.
\end{document}