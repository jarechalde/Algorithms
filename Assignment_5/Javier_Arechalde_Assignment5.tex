\documentclass{article}
\usepackage{listings}
\usepackage{amsmath}
\usepackage{graphicx}
\usepackage{float}
\usepackage{subcaption}
\usepackage[linewidth=1pt]{mdframed}
\usepackage[colorlinks]{hyperref}

\usepackage{verbatim}

\usepackage{algorithm}
\usepackage{algpseudocode}

\hypersetup{citecolor=DeepPink4}
\hypersetup{linkcolor=DarkRed}
\hypersetup{urlcolor=Blue}

\usepackage{cleveref}

\setlength{\parindent}{1em}
\setlength{\parskip}{1em}
\renewcommand{\baselinestretch}{1.0}

\begin{document}

\begin{titlepage}
	\centering
	{\scshape\LARGE Assignment 4\par}
	\vspace{1cm}
	{\scshape\Large Algorithms \& Complexity (CIS 522-01)\par}
	\vspace{1.5cm}
	{\Large\itshape Javier Arechalde\par}
	\vfill
	{\large \today\par}
\end{titlepage}

\section*{Part A: Read the solved exercises and Practice}

\subsection*{Solved excercise \#1 in Chapter 6}

In this problem, we want to place billboards in a highway to get maximum revenue. The highway will be $M$ miles long, and will have $n$ locations on which we can locate the different billboards, each one of this locations will give us $r_i>0$ revenue. There is also a regulation that doesn't allow two billboards to be placed closer than 5 miles away from each other.

The goal of this problem is to find the billboard placements that will give us the maximum revenue, while following all the given regulations.

\subsubsection*{Algorithm Pseudocode}

\begin{algorithm}[H]
\caption{Implementation}
\begin{algorithmic}[1]
\State Initiallize $M[0] = 0$ and $M[1] = r_1$
\For{$j = 2,3,...,n$}
 \If{$x_j-x_{j-1} \geq 5$}
  \State $M[j] = M[j-1] + r_j$
 \Else { Find the closest possible value ($x_j-x_i \geq 5$)}
  \If{$M[i]+r_{j}>M[j-1]$}
   \State $M[j] = M[i] + r_j$
  \Else
   \State $M[j] = M[j-1]$
  \EndIf
 \EndIf
\EndFor
\State
\State \Return M[n]
\end{algorithmic}
\end{algorithm}

\subsubsection*{Solution for problem instance of size $10$}

The code for a problem instance of size 10 is as follows. Run it to see the results.

\lstinputlisting[language = Python]{Highway.py}

\subsubsection*{Time Complexity}


\section*{Part B: Problem Solving}

%Chapter 6, Exercise 2

\subsection*{Consulting Jobs}

\subsubsection*{Problem Model}

\subsubsection*{Pseudocode}

\begin{algorithm}[H]
\caption{Implementation}
\begin{algorithmic}[1]
\end{algorithmic}
\end{algorithm}

\subsubsection*{Implementation}

Here is the code for the implementation of the \textit{pseudocode} shown below.

\lstinputlisting[language=Python]{}

\subsubsection*{Running time}
As we have to run over all the values in the set, to find the optimal combination of billboard locations, the running time of our implementation will be $O(n)$.

%Chapter 6, Exercise 11

\subsection*{Carrier Selection}

\subsubsection*{Problem Model}

\subsubsection*{Pseudocode}

\begin{algorithm}[H]
\caption{Implementation}
\begin{algorithmic}[1]
\end{algorithmic}
\end{algorithm}

\subsubsection*{Running time}

\end{document}
