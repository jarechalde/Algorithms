\documentclass{article}
\usepackage{listings}
\usepackage{amsmath}
\usepackage{graphicx}
\usepackage{float}
\usepackage{subcaption}
\usepackage[linewidth=1pt]{mdframed}
\usepackage[colorlinks]{hyperref}

\usepackage{verbatim}

\usepackage{algorithm}
\usepackage{algpseudocode}

\hypersetup{citecolor=DeepPink4}
\hypersetup{linkcolor=DarkRed}
\hypersetup{urlcolor=Blue}

\usepackage{cleveref}

\setlength{\parindent}{1em}
\setlength{\parskip}{1em}
\renewcommand{\baselinestretch}{1.0}

\begin{document}

\begin{titlepage}
	\centering
	{\scshape\LARGE Assignment 4\par}
	\vspace{1cm}
	{\scshape\Large Algorithms \& Complexity (CIS 522-01)\par}
	\vspace{1.5cm}
	{\Large\itshape Javier Arechalde\par}
	\vfill
	{\large \today\par}
\end{titlepage}

\section*{Part A: Read the solved exercises and Practice}

\subsection*{Solved excercise \#1 in Chapter 6}

In this problem, we want to place billboards in a highway to get maximum revenue. The highway will be $M$ miles long, and will have $n$ locations on which we can locate the different billboards, each one of this locations will give us $r_i>0$ revenue. There is also a regulation that doesn't allow two billboards to be placed closer than 5 miles away from each other.

The goal of this problem is to find the billboard placements that will give us the maximum revenue, while following all the given regulations.

\subsubsection*{Algorithm Pseudocode}

\begin{algorithm}[H]
\caption{Implementation}
\begin{algorithmic}[1]
\State Initiallize $M[0] = 0$ and $M[1] = r_1$
\For{$j = 2,3,...,n$}
 \If{$x_j-x_{j-1} \geq 5$}
  \State $M[j] = M[j-1] + r_j$
 \Else { Find the closest possible value ($x_j-x_i \geq 5$)}
  \If{$M[i]+r_{j}>M[j-1]$}
   \State $M[j] = M[i] + r_j$
  \Else
   \State $M[j] = M[j-1]$
  \EndIf
 \EndIf
\EndFor
\State
\State \Return M[n]
\end{algorithmic}
\end{algorithm}

\subsubsection*{Solution for problem instance of size $10$}

The code for a problem instance of size 10 is as follows. Run it to see the results.

\lstinputlisting[language = Python]{Highway.py}

\subsubsection*{Time Complexity}
As we have to run over all the values in the set, to find the optimal combination of billboard locations, the running time of our implementation will be $O(n)$.

\section*{Part B: Problem Solving}

%Chapter 6, Exercise 2

\subsection*{Consulting Jobs}

We are in a company that does consulting jobs. Each week we have two different jobs to choose from, one thats a low stress job and another one thats a high stress job. Each one of this jobs will give us a different revenue $l_i$ for the low stress job and $h_i$ for the high stress job. The requirement for choosing a high stress job is that the previous week the team should have been resting, then doing no job at all. 

The goal is to find the combination of jobs that gives us the maximum revenue, taking into account that if we want to choose a high stress job, the condition stated before has to be met.

\subsubsection*{Problem Model}

We will solve this problem using dynamic programming. In each iteration we will have to choose between doing the high stress job, doing the low stress job, or resting so the next day we can do the high stress job.

We will have a list of low stress jobs, and it's respective revenue for each of them $l = [l_1 = rl_1,l_2 = rl_2,...,l_n = rl_n]$ and a list of the high stress jobs $h = [h_1 = rh_1, h_2 = rh_2,...,h_n = rh_n]$ with these jobs, and the revenue expected for each of them.

We will return a list containing the planning, and the expected revenue from this planning.

\subsubsection*{Pseudocode}

\begin{algorithm}[H]
\caption{Implementation}
\begin{algorithmic}[1]
\For{iteration $i = 1$ to $n$}
 \If{$h_{i+1}>l_i+l_{i+1}$}
  \State We choose to rest in week $i$
  \State We choose the high stress job on week $i+1$
  \State Continue on iteration $i+2$
 \Else
  \State We choose low-stress job
  \State We continue on iteraton $i+1$
 \EndIf
\EndFor
\end{algorithmic}
\end{algorithm}

\subsubsection*{Implementation}

Here is the code for the implementation of the \textit{pseudocode} shown below.

\lstinputlisting[language=Python]{Consulting.py}

\subsubsection*{Running time}

The running time of our algorithm will be $O(n)$, as we have to go over all the $n$ weeks to choose the best possible combination of jobs.

%Chapter 6, Exercise 11

\subsection*{Carrier Selection}

\subsubsection*{Problem Model}

\subsubsection*{Pseudocode}

\begin{algorithm}[H]
\caption{Implementation}
\begin{algorithmic}[1]
\end{algorithmic}
\end{algorithm}

\subsubsection*{Running time}

\end{document}
