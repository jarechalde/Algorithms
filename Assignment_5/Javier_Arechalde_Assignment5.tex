\documentclass{article}
\usepackage{listings}
\usepackage{amsmath}
\usepackage{graphicx}
\usepackage{float}
\usepackage{subcaption}
\usepackage[linewidth=1pt]{mdframed}
\usepackage[colorlinks]{hyperref}

\usepackage{verbatim}

\usepackage{algorithm}
\usepackage{algpseudocode}

\hypersetup{citecolor=DeepPink4}
\hypersetup{linkcolor=DarkRed}
\hypersetup{urlcolor=Blue}

\usepackage{cleveref}

\setlength{\parindent}{1em}
\setlength{\parskip}{1em}
\renewcommand{\baselinestretch}{1.0}

\begin{document}

\begin{titlepage}
	\centering
	{\scshape\LARGE Assignment 4\par}
	\vspace{1cm}
	{\scshape\Large Algorithms \& Complexity (CIS 522-01)\par}
	\vspace{1.5cm}
	{\Large\itshape Javier Arechalde\par}
	\vfill
	{\large \today\par}
\end{titlepage}

\section*{Part A: Read the solved exercises and Practice}

\subsection*{Solved excercise \#1 in Chapter 6}

In this problem, we want to place billboards in a highway to get maximum revenue. The highway will be $M$ miles long, and will have $n$ locations on which we can locate the different billboards, each one of this locations will give us $r_i>0$ revenue. There is also a regulation that doesn't allow two billboards to be placed closer than 5 miles away from each other.

The goal of this problem is to find the billboard placements that will give us the maximum revenue, while following all the given regulations.

\subsubsection*{Algorithm Pseudocode}

\begin{algorithm}[H]
\caption{Billboards Pseudocode}
\begin{algorithmic}[1]
\State Initiallize $M[0] = 0$ and $M[1] = r_1$
\For{$j = 2,3,...,n$}
 \If{$x_j-x_{j-1} \geq 5$}
  \State $M[j] = M[j-1] + r_j$
 \Else { Find the closest possible value ($x_j-x_i \geq 5$)}
  \If{$M[i]+r_{j}>M[j-1]$}
   \State $M[j] = M[i] + r_j$
  \Else
   \State $M[j] = M[j-1]$
  \EndIf
 \EndIf
\EndFor
\State \Return M[n]
\end{algorithmic}
\end{algorithm}

\subsubsection*{Solution for problem instance of size $10$}

The code for a problem instance of size 10 is as follows. Run it to see the results.

\lstinputlisting[language = Python]{Highway.py}

\subsubsection*{Time Complexity}
As we have to run over all the values in the set, to find the optimal combination of billboard locations, the running time of our implementation will be $O(n)$.

\section*{Part B: Problem Solving}

%Chapter 6, Exercise 2

\subsection*{Consulting Jobs}

We are in a company that does consulting jobs. Each week we have two different jobs to choose from, one thats a low stress job and another one thats a high stress job. Each one of this jobs will give us a different revenue $l_i$ for the low stress job and $h_i$ for the high stress job. The requirement for choosing a high stress job is that the previous week the team should have been resting, then doing no job at all. 

The goal is to find the combination of jobs that gives us the maximum revenue, taking into account that if we want to choose a high stress job, the condition stated before has to be met.

\subsubsection*{Problem Model}

We will solve this problem using dynamic programming. In each iteration we will have to choose between doing the high stress job, doing the low stress job, or resting so the next day we can do the high stress job.

We will have a list of low stress jobs, and it's respective revenue for each of them $l = [l_1 = rl_1,l_2 = rl_2,...,l_n = rl_n]$ and a list of the high stress jobs $h = [h_1 = rh_1, h_2 = rh_2,...,h_n = rh_n]$ with these jobs, and the revenue expected for each of them.

We will first initialize the maximum revenue for week 0 to 0. We suppose that the team is rested on the first week, then the maximum revenue on week 1 will be the greater between the low stress job and the high stress job.

Then we will iterate through all weeks, if the revenue that we will get from resting the week before and doing the high stress is greater than doing that week's and the week before low stress job, then we will rest on the week before, and therefore add to the maximum revenue two weeks before the revenue we will get from doing the high stress job. Otherwise, we won't rest the week before, so we will add the low stress job value on that week to the previous weeks' maximum revenue.

We will return a list containing the maximum possible revenue for each of the weeks.

\subsubsection*{Pseudocode}

\begin{algorithm}[H]
\caption{Consulting Jobs Pseudocode}
\begin{algorithmic}[1]
\State We initiallize $Revenue_0 = 0$
\State We initialize $Revenue_1 = max(l_1,h_1)$
\For{iteration $i = 2 \to n$}
 \If{$h_{i}>l_i+l_{i-1}$}
  \State We decide to rest the week before, then
  \State $Revenue_i = Revenue_{i-2}+h_i$
 \Else
  \State $Revenue_i = Revenue_{i-1}+l_i$
 \EndIf
\EndFor
\State \Return Best possible revenue $Revenue_n$
\end{algorithmic}
\end{algorithm}

\subsubsection*{Implementation}

Here is the code for the implementation of the \textit{pseudocode} shown below.

\lstinputlisting[language=Python]{Consulting.py}

\subsubsection*{Running time}

The running time of our algorithm will be $O(n)$, as we have to go over all the $n$ weeks to choose the best possible combination of jobs.

%Chapter 6, Exercise 11

\subsection*{Carrier Selection}

We are a consulting company that manufactures PC equipment and ships it all over the country. We have a projected supply $s_i$ for the next weeks, which is measured in pounds. Each week's suppy can be carried by two different companies.

\begin{itemize}
 \item Company A: Charges per pound $r$
 \item Company B: Charges a fixed amount $c$ per week
\end{itemize}

The only restriction in this problem is that if we select company B, it must be chosen in four continuous weeks at a time.

Our goal is to find the combination of carriers that results in the minimum shipping cost.

\subsubsection*{Problem Model}

In this problem, we will have an array $s$ containing the projected supply for the next $n$ weeks, we will also have $r$ which is the amount that company A charges per pound of the produced PC equipment, and $c$ that is the fixed amount company B charges per week for delivering the equipment.

At the beginning of our algorithm, we will start by only choosing company A, until we reach week 4, once we reach week 4, we will start checking  what is more expensive, if choosing company A for the past 3 weeks and that week, or choosing the flat rate of company B. In case using company B is the better choice, we will update the best cost for that week with the best cost four weeks ago plus the flat rate multiplied by the four weeks. Otherwise we will add the cost per pound multiplied for the projected supply for that week to the previous week best cost.

In the end, we will have an array containing the best cost possible for each of the weeks.

\subsubsection*{Pseudocode}

\begin{algorithm}[H]
\caption{Carrier Selection Pseudocode}
\begin{algorithmic}[1]
\State We set $Cost_0 = 0$
\For{$i$ in range $0 \to n$}
 \If{$i<3$}
  \State $Cost_i = Cost_{i-1}+s_i*r$
 \Else
  \If{$Cost_{i-1}+s_i*r>4*c$}
   \State $Cost_i = Cost_{i-4}+4*c$
  \Else
   \State $Cost_i = Cost_{i-1}+s_i*r$
  \EndIf
 \EndIf
\EndFor
\State \Return Best cost possible $Cost_n$
\end{algorithmic}
\end{algorithm}

\subsubsection*{Running time}

The running time of our implementation will be $O(n)$, as we have to go through the $n$ weeks to find the lowest cost possible to deliver our product supply.

\end{document}
