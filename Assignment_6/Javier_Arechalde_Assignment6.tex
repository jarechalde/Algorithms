\documentclass{article}
\usepackage{listings}
\usepackage{amsmath}
\usepackage{graphicx}
\usepackage{float}
\usepackage{subcaption}
\usepackage[linewidth=1pt]{mdframed}
\usepackage[colorlinks]{hyperref}

\usepackage{verbatim}

\usepackage{algorithm}
\usepackage{algpseudocode}

\hypersetup{citecolor=DeepPink4}
\hypersetup{linkcolor=DarkRed}
\hypersetup{urlcolor=blue}

\usepackage{cleveref}

\setlength{\parindent}{1em}
\setlength{\parskip}{1em}
\renewcommand{\baselinestretch}{1.0}

\begin{document}

\begin{titlepage}
	\centering
	{\scshape\LARGE Assignment 6\par}
	\vspace{1cm}
	{\scshape\Large Algorithms \& Complexity (CIS 522-01)\par}
	\vspace{1.5cm}
	{\Large\itshape Javier Arechalde\par}
	\vfill
	{\large \today\par}
\end{titlepage}

\section*{1. Ford-Fulkerson Algorithm}

\subsection*{a. Implementation}

We used the Ford-Fulkerson implementation found \href{https://www.geeksforgeeks.org/ford-fulkerson-algorithm-for-maximum-flow-problem/}{here}.

This implementation models the graph as an Adjacency Matrix, and then runs Ford-Fulkerson algorithm over this graph, using Breadth First Search (BFS) to find the augmenting path in the graph.

\begin{algorithm}[H]
\caption{Ford-Fulkerson Pseudocode}
\begin{algorithmic}[1]
\Function{Ford-Fulkerson}{}
 \State We initally set the flow to 0 for all the edges in the graph
 \While{There is a path in the residual graph}
  \State We find the minimum residual capacity of the edges along the path
  \State Increment the flow by that minimum residual capacity $path_{flow}$
  \State $max_{flow} = max_{flow} + path_{flow}$
  \State We update the residual graph too
 \EndWhile
 \State \Return Maximum flow $\rightarrow max_{flow}$
\EndFunction
\end{algorithmic}
\end{algorithm}

\subsection*{b. Time Complexity}

\subsection*{c. Results}

The maximum possible flow found is 36.

\section*{2. Project Selection Problem}

\section*{3. Doctor Holiday Assignment}

\section*{4. Advertisement Problem}

\end{document}
