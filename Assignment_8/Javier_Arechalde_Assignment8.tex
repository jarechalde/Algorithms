\documentclass{article}
\usepackage{listings}
\usepackage{amsmath}
\usepackage{graphicx}
\usepackage{float}
\usepackage{subcaption}
\usepackage[linewidth=1pt]{mdframed}
\usepackage[colorlinks]{hyperref}

\usepackage{verbatim}

\usepackage{algorithm}
\usepackage{algpseudocode}

\hypersetup{citecolor=DeepPink4}
\hypersetup{linkcolor=DarkRed}
\hypersetup{urlcolor=blue}

\usepackage{cleveref}

\setlength{\parindent}{1em}
\setlength{\parskip}{1em}
\renewcommand{\baselinestretch}{1.0}

\begin{document}

\begin{titlepage}
	\centering
	{\scshape\LARGE Assignment 7\par}
	\vspace{1cm}
	{\scshape\Large Algorithms \& Complexity (CIS 522-01)\par}
	\vspace{1.5cm}
	{\Large\itshape Javier Arechalde\par}
	\vfill
	{\large \today\par}
\end{titlepage}

\section*{1. File transfer}

\subsection*{Problem Model}

In this problem we will have $k$ network sources from source to destination, and $n$ files that we want to transfer from source to destination. Each one of the files $f_i$ has a length $l_i$. We want to find out how to assign the files to each one of the $k$ routes so we can transfer the $n$ files in the shortest time.

We will also assume in this problem that the network capacity is unlimited for each one of the networks.

\subsection*{Class}

This algorithm belongs to the P class, as it can be implemented as an approximation greedy algorithm for load balancing.

\subsection*{Algorithm}

In this algorithm, we will first sort the file lengths in decreasing order, and then we will start scheduling each one of the files to the less loaded network at each step. We will terminate this execution whenever we schedule all the files.

\subsection*{Pseudocode}

\begin{algorithm}[H]
\caption{File tranfers pseudocode}
\begin{algorithmic}[1]
\State We first sort all the files $f_i$ in decreasing order
\While{We didn't assign all the files}
 \State Assign file $f_i$ to the least loaded network $N_i$
\EndWhile
\State \Return The list of the list assignments
\end{algorithmic}
\end{algorithm}

\subsection*{Problem instance}

An example of how to solve this problem can be found in the image attached to this assignment \textit{Problem1.jpg}.

\subsection*{Time complexity}

The time complexity of this implementation is $O(n\log m)$, as when we want to schedule each one of the jobs, we have to find which is the machine, or in this case the network, with the least load.

\subsection*{Approximation guarantee}

The approximation guarantee for load balancing with list-scheduling is:

$$L\leq \frac{3}{2} L^{\ast}$$

\section*{2. Multiple Interval Scheduling}

\subsection*{Problem Model}

\subsection*{Class}

\section*{3. Airport service}

\subsection*{Problem Model}

In this problem we have $n$ airport sites, and $m$ direct flights scheduled between these airports. Building a service facility that allow us to operate in some airport $A_i$ has a cost $C_i$. We want to select a subset of airports in such a way that we can have all airports connected, at the minimum cost possible.

We will model this problem as a directed graph $G$, in which the airports will be the nodes, and the flights will be the edges. Also, each one of this nodes will have a cost associated $C_i$ that will be the cost of building the service facility in that airport $A_i$.

\subsection*{Class}

This problem belongs to class $P$, because it can be solved by an approximation method using a \textit{greedy} approach.

\subsection*{Algorithm}

To solve this problem, we will use a greedy vertex cover approach, or as it is usually named, the \textit{pricing method}.

In this algorithm, we will first set the price paid for every edge to $0$, and then , we will continuously increase the edges values, untill one of the nodes at one of the sides of the edge is 'tight', at that point we will add that node to the set of tight nodes $S$. We will then continue to look for edges that are not connected to tight nodes, to increase the edge value until one of the nodes that the edge is connected to becomes tight. We will continue to do this until we can't find an edge which is not connected to any tight node. Then we will return the set of tight nodes, which will be the solution to our problem. In our case the nodes returned would be the airports in which we should build the service facilities.

\subsection*{Pseudocode}

\begin{algorithm}[H]
\caption{Pricing method pseudocode}
\begin{algorithmic}[1]
\State We first set $p_e = 0$ for all edges $e\in E$
\While{There is an edge $e=(i,j)$ such that neither $i$ nor $j$ is tight}
 \State We selet an edge $e$
 \State We increase $p_e$ without violating fairness
\EndWhile
\State \Return $S$ which is the set of all tight nodes
\end{algorithmic}
\end{algorithm}

\subsection*{Problem instance}

A small problem instance of this algorihtm can be found in \textit{Figure 11.8} in Algorithm Design book page 622.

\subsection*{Time complexity}

The time complexity of this algorithm will be $O(m)$, where $m$ is the number of edges, or number of direct flights. As in the worst case scenario, we will have to go through all edges to find all the tight nodes.

\subsection*{Approximation guarantee}

The approximation guarantee for this algorithm is:

$$w(S)\leq 2\sum_{e\in E}^{} p_e\leq 2w(S^{\ast})$$

\section*{4. Telecommunication company}

\subsection*{Problem Model}

In this problem, we have to build base stations in various locations to provide mobile phone service. We have houses located in $n$ different locations. We want to select $k$ locations to build $k$ base stations so that the maximum distance from any ouse to its closest station is minimized.

\subsection*{Class}

This algorithm belongs to $P$ class, as it can be implemented as an approximation algorithm with a greedy approach.

\subsection*{Algorithm}

In this algorithm, we will start by setting one of the houses in the $n$ locations to buold the base station, then we will build the next location on the house that is further from the set of centers or base stations. We will continue doing this until we can't place any more base stations.

\subsection*{Pseudocode}

\begin{algorithm}[H]
\caption{Center selection pseudocode}
\begin{algorithmic}[1]
\State We assume that the number of centers is fewer that the number of points
\State Otherwise, our centers will be placed in top of our locations
\State We first select any site s and $C = {s}$
\While{$|C|<k$}
 \State Select a site $s\in S$ that maximizes $dist(s,C)$
 \State Add site $s$ to $C$
\EndWhile
\State \Return $C$ as the selected set of sites
\end{algorithmic}
\end{algorithm}

\subsection*{Problem instance}

An example of how to solve this problem can be found in the image attached to this assignment \textit{Problem4.jpg}.

\subsection*{Time complexity}

The time complexity of this algorithm will be $O(k)$, where $k$ is the number of points that we are looking to place.

\subsection*{Approximation guarantee}

This greedy algorithm returns a set $C$ of $k$ points such that $r(C)\leq 2r(C\ast)$, where $C\ast$ is an optimal set of $k$ points.

\section*{5. Job requests}

\subsection*{Problem Model}

In this problem, we have $n$ job requests to select from, each job paying $p_i$ amount of reward. Some of these jobs can't be scheduled together due to time conflicts. Our goal is to select the set of jobs that have no conflict among themselves that gives us the highest total amount of reward.

We will model this problem as a tree, in which the nodes are the jobs, and the edges are the conflicts. Each one of the nodes (jobs) will have associated a certain amount, that will be the reward for completing that job. Then our goal will be to find the maximum-weight independent set on the tree.

\subsection*{Class}

The problem of finding a maximum weight independent set in a tree belongs to $P$ class.

\subsection*{Algorithm}

We will first root an arbitrary node $r$, to orientate all the tree's edges away from $r$. For a node $u$ the nodes that are in between this node and $r$ will be then the parent nodes, and the nodes that face the other way will be the children nodes.

Then we will use a dynamic programming solution at each node, in which we will have to choose to include that node or not depending on which gives the maximum revenue, if including that node, but not its children, or including the children nodes but not that node.

\subsection*{Pseudocode}

\begin{algorithm}[H]
\caption{Find maximum-weight independent set pseudocode}
\begin{algorithmic}[1]
\State First root the tree at arbitrary node $r$
\For{All nodes $u\in T$ in post-order}
 \If{$u$ it's a leaf node}
  \State $M_{out}[u] = 0$
  \State $M_{in}[u] = w_u$
 \Else{$u$ it's not a leaf node}
  \State $M_{out}[u] = \sum_{v\in children(u)}^{} max(M_{out}[u],M_{in}[u])$
  \State $M_{in}[u] = w_u + \sum_{v\in children(u)}^{} M_{out}[u]$
 \EndIf
\EndFor
\Return $max(M_{out}[r],M_{in}[r])$
\end{algorithmic}
\end{algorithm}

\subsection*{Problem instance}

An example of how to solve this problem can be found in the image attached to this assignment \textit{Problem5.jpg}.

\subsection*{Time complexity}

The time complexiy of this algorithm will be $O(n)$, where $n$ is the number of nodes, in this case the number of jobs.

\section*{6. City roads monitoring}

\subsection*{Problem Model}

We will model this problem as a graph $G$, in which the nodes will be the crossroads, and the edges will be the roads.

\subsection*{Class}

As $k$ is small in this problem, we can solve this vertex cover problem in polynomial time. Then this problem belongs to P.

\subsection*{Algorithm}

Our algorithm will check all the possible subsets of $V$ of size $k$, and check if any of them is a vertex cover. This algorithm implementation will be based on brute force.

\subsection*{Pseudocode}

\begin{algorithm}[H]
\caption{Vertex cover pseudocode}
\begin{algorithmic}[1]
\While{There is a subset $S$ of $V$ of size $k$ that we didn't check yet}
 \State We check if subset $S$ is a vertex cover in our graph
 \If{It's a vertex cover}
  \State \Return Vertex cover $S$
  \State Break the loop
 \ElsIf{It's not a vertex cover}
  \State We continue with the next possible subset
 \EndIf
\EndWhile
\end{algorithmic}
\end{algorithm}

\subsection*{Problem instance}

An example of how to solve this problem can be found in the image attached to this assignment \textit{Problem6.jpg}.

\subsection*{Time complexity}

There are $\binom{n}{k}$ possible subsets, and as it takes $O(kn)$ time to check wether a subset is a vertex cover or not, the time complexity of our algorithm will be $O(kn^{k+1})$

\end{document}
